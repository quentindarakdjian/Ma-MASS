\chapter*{Introduction générale}
\addcontentsline{toc}{chapter}{Introduction générale}

\section*{Contexte}

La crise économique des années 2010 touche l'ensemble de l'économie occidentale et s'accompagne de la dégradation de plusieurs fondamentaux sociaux et environnementaux de notre société. Cette situation nous oblige à repenser nos modèles de développement et en particulier à nous interroger sur l'avenir de l'énergie qui fonde le développement du monde que l'on connait. Comme le confirme à chaque rapport le GIEC \cite{GIEC-14}, abondante et bon marché depuis la seconde révolution industrielle, l'énergie fossile apparaît aujourd'hui et depuis les chocs pétroliers comme rare et de plus en plus chère dans un contexte où la menace du changement climatique se renforce. Accentué par l'évolution démographique, c'est dans ce contexte que le controversé essayiste et scientifique Rifkin \cite{Rifkin-12} promeut la troisième révolution industrielle et considère la transition vers des systèmes énergétiques décarbonés comme prioritaires. Néanmoins la démarche de transition énergétique (Energiewende) est à mettre au crédit de l'Allemagne de 1980 qui a été le précurseur mondial de l'application d'énergie renouvelable à l'échelle industrielle. En France et à la suite du Grenelle de l'Environnement, cette transition énergétique est partagée par la grande majorité des politiques, afin d'atteindre l'engagement national du facteur 4 \footnote{Le facteur 4 en France correspond à une réduction des émissions par quatre des GES entre 1990 et 2050} en 2050, tout en restant compétitif sur les marchés internationaux. L'association Negawatt \footnote{La sémantique du terme Négawatt quantifie une puissance en moins, c'est-à-dire une puissance économisée par un changement de technologie ou de comportement} a une forte notoriété pour la mise en œuvre de la transition énergétique de manière concrète et plus technique que le travail de Rifkin. L'accord de Paris de 2015 signé suite aux négociations de la COP21, est le dernier signe d'une volonté internationale de réduire le recours aux énergies fossiles et de réduire l'émission de gaz à effet de serre.

Cette contrainte de réduction des consommations d'énergie confronte nos sociétés à un défi d'une ampleur considérable. La consommation d'énergies fossiles à grande échelle est le socle qui a rendu possible le développement de nos sociétés modernes où aucune activité humaine n'échappe à la consommation d'énergie. En occident, le premier secteur consommateur d'énergie est le bâtiment (de 30 à 40 \% selon les pays), suivi des transports et de la production industrielle. Pour parvenir à l'objectif d'une société plus sobre en énergie, le secteur du bâtiment est donc une priorité tant son potentiel est jugé important et accessible à moyen terme. Compte tenu de l'état des technologies et par rapport aux transports, le bâtiment apparaît comme le domaine le plus mature pour la transition énergétique.

Cette amélioration continue des performances énergétiques des bâtiments neufs et anciens a été accompagnée par le développement d'outils de plus en plus performants et précis en termes de modélisation numérique. A l'inverse des premiers outils de calcul des déperditions statiques ne tenant même pas compte des apports solaires, les outils de simulations thermiques dynamiques actuels permettent d'intégrer l'ensemble des paramètres influençant le fonctionnement d'un bâtiment: climat, physique du bâtiment, équipements et usages. Alors que le niveau d'incertitude concernant les paramètres statiques liés à l'enveloppe, tels que les propriétés des matériaux ou le contrôle qualité de chantier, sont de plus en plus faibles, les incertitudes sur l'utilisation effective du bâtiment et le comportement des utilisateurs sont quant à elles très importantes. En résulte dans la pratique des écarts parfois considérables entre consommations théoriques et mesurées, d'autant plus élevés que la performance du bâtiment est grande. Au travers de l'amélioration des modèles des bâtiments, cette thèse doit permettre de mieux prendre en considération le comportement des usagers notamment par un état de l'art d'études sociologiques appliquées à l'énergétique du bâtiment et de développer un module de simulation pour le bureau d'études, AI Environnement, la structure d'accueil.

\section*{Force collective}

Comme nous venons de le voir, les pouvoirs publics ainsi que les associations non gouvernementales prennent des mesures pour panser les maux de notre monde. Certes, les moyens ne semblent pas toujours à la hauteur des enjeux, néanmoins les comportements individuels peuvent mener à des améliorations significatives. Cette philosophie a été imagée par l'histoire du mouvement Colibris de Pierre Rabhi: "Un jour, il y eut un immense incendie dans la forêt, seul un colibri déposa, goutte après goutte, de l'eau sur les arbres. "Tu crois que ce sont tes gouttes d'eau qui vont arrêter l'incendie?", se moquèrent les autres oiseaux. Et le colibri de répondre: "Seul, non, mais j'aurais fait ma part." Le projet de la tour Elithis de Dijon, se voulant être le premier bâtiment tertiaire à énergie positive, est un exemple d'application du mouvement Colibris. Lors de sa mise en service la tour ne consommait que 20 $kWh/m^2/an$, soit six fois moins qu'un bâtiment tertiaire standard, mais toujours 20 $kWh/m^2/an$ de trop pour véritablement atteindre l'objectif. Or, ce n'est pas un acharnement technologique qui a permis cela, mais l'accompagnement des employés vers des comportements en adéquation avec ce bâtiment et son environnement.

Colibris et les mouvements semblables sont des accélérateurs de transition, en s'appuyant sur la capacité de chacun à changer et à incarner ce changement dans des expériences concrètes et collectives. Cela encourage l'émergence et l'incarnation de nouveaux modèles de société fondés sur l'autonomie, l'écologie et l'humanisme.

\textit{"Les Colibris, ce sont tous ces individus qui inventent, expérimentent et coopèrent concrètement, pour bâtir des modèles de vie en commun, respectueux de la nature et de l'être humain."} P. Rabhi

\section*{Thèse en entreprise}

Le travail présenté dans ce manuscrit est le produit d'une thèse en contrat CIFRE (Convention Industrielle de Formation par la Recherche) c'est à dire co-financée par une entreprise, supervisée par un laboratoire et subventionnée par l'ANRT (Association Nationale de la Recherche et de la Technologie). Dans une démarche d'innovation et pour structurer ses activités de recherche et développement, AI Environnement, le bureau d'études techniques et Antoine Boulla, ingénieur d'étude et encadrant principal de la thèse, ont ressenti le besoin de mieux appréhender le comportement des usagers pour leurs études énergétiques. C'est dans ce contexte que la synergie avec l'Université de La Rochelle, par l'entremise de Christian Inard et Jean-Marc Ogier respectivement du LaSIE (Laboratoire des Sciences de l'Ingénieur pour l'Environnement) et L3I (Laboratoire Informatique, Image et Interaction) s'est établie. Afin de pouvoir confronter les résultats du modèle à des données réelles, Bassam Moujalled, membre du CEREMA \footnote{CEREMA: Centre d'Etudes et d'expertise sur les Risques, l'Environnement, la Mobilité et l'Aménagement \url{http://www.cerema.fr/}}, a été greffé dés le début du projet de thèse pour sa connaissance de la modélisation dynamique des bâtiments et sa proximité à des projets expérimentaux tests. Suite au départ d'Antoine Boulla d'AI Environnement en milieu de thèse, Sylvain Bille co-fondateur de l'entreprise en 2008 a repris l'encadrement. Enfin, à la suite d'un Master en Sciences et Techniques des Environnements Urbains à l'Ecole des Mines de Nantes, j'ai complété le groupe de travail afin de réaliser et articuler le projet.

La thèse en contrat CIFRE s'inscrit dans une logique de don contre-don où, par l'intermédiaire du doctorant, AI Environnement s'est lié à l'expérience de laboratoires de l'Université de La Rochelle tandis que les laboratoires ont recadré leurs activités de recherches aux besoins de la structure industrielle. Au milieu de cette synergie, le doctorant se retrouve parfois en situation inconfortable où il doit trouver l'équilibre entre les attentes des différentes parties tout étant l'acteur et le décideur principal du projet.

Au travers d'une telle thèse, le doctorant a un rôle d'interface entre le monde universitaire et le monde industriel. Bien que certains recruteurs voient encore la formation doctorale comme non professionnalisante, la CIFRE est vue d'une autre manière et tend à faire évoluer les mentalités. Les recruteurs industriels trouvent chez ces docteurs des compétences de pointe, d'une part de chercheur et d'autre part d'ingénieur familiarisé au milieu professionnel associé: double compétence valorisable.

\section*{Enjeux de la thèse et contribution}

C'est donc dans un contexte partagé entre le bureau d'études et les laboratoires, que cette thèse contribue à améliorer les résultats des projets thermiques et environnementaux. La recherche de hautes performances énergétiques et environnementales amène les bureaux d'études techniques à réaliser des STD (Simulations Thermiques Dynamiques) afin d'optimiser les constructions et rénovations en diminuant leurs consommations et en améliorant le confort des occupants. Le constat est que la prise en compte du comportement de ces derniers est actuellement le maillon faible de ce genre d'études. Face au manque d'informations concernant le comportement réel des usagers et leurs interactions avec le bâtiment, les thermiciens et énergeticiens sont contraints à en simplifier les hypothèses d'entrée. En effet, aujourd'hui ce comportement humain est grossièrement réduit à un taux d'occupation, une présence journalière répétitive via une approche très déterministe.

De nombreuses raisons ont été avancées pour expliquer la non-précision et la non-fiabilité des résultats des études de STD. Cependant, les résultats de l'Annexe 53 \cite{Annex-53-1}\footnote{Présentation et publications: \url{http://www.iea-ebc.org/index.php?id=141}}, \textit{Total Energy Use in Buildings: Analysis and Evaluation Methods}, projet de l'Agence Internationale de l'Energie (AIE), montrent que la principale cause d'incertitude provient de la mauvaise prise en compte du comportement des occupants. Dans la continuité à ce projet, l'Annexe 66, \textit{Definition and Simulation of Occupant Behavior in Buildings}\footnote{Site internet: \url{http://www.annex66.org/}} a été lancée avec des objectifs et des problématiques de recherche qui coïncident à ceux de la thèse, à savoir:
\begin{itemize}
\item Identification et modélisation des comportements impactant les besoins énergétiques des bâtiments
\item Implémentation des modèles dans les outils de simulation énergétique des bâtiments
\item Évaluation des modèles de comportement développés sur des cas d'études
\end{itemize}

\section*{Plan du manuscrit}

Nous proposons une lecture du manuscrit qui peut être découpée en plusieurs parties. Le premier chapitre évoque la situation de la conception et de la construction des bâtiments en France. Le second chapitre traite de la différence de performance théorique et réelle...