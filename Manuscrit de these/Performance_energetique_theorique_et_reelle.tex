\chapter{Performance énergétique théorique et réelle}

Nous venons d'évoquer le contexte global de la conception et de la construction sous l'angle d'un bureau d'études thermique et environnement. Le travail de ce type de structure est de donner un avis d'expert sur les questions énergétiques. Pour cela, les outils numériques comme les Simulations Thermiques Dynamiques (STD) sont utilisés afin de modéliser les futures constructions et prédire leurs performances.

Or des doutes émanent sur la fidélité des modèles à reproduire les phénomènes énergétiques des bâtiments. Cette différence entre ce qui est prédit et ce qui est mesuré n'est généralement pas anecdotique, c'est ce que les anglophones appellent le \textit{performance gap}. Les performances des bâtiments sont influencées par de nombreux facteurs, tels que les conditions climatiques, les équipements et la structure du bâtiment, mais aussi par des facteurs relatifs aux occupants eux mêmes, comme leurs comportements, leurs souhaits de conditions environnementales intérieures et la maintenance de leurs équipements. Les effets de la modélisation du climat, de l'enveloppe et des systèmes sont relativement bien connus et standardisés, alors que ceux en lien à l'utilisation du bâtiment se révèlent être plus incertains. On comprend alors que prédire les consommations totales des bâtiments est un exercice très complexe si l'on considère l'ensemble de ces éléments.

Ce travail de simplification de la réalité doit néanmoins être réalisé pour quantifier et garantir les performances des travaux de rénovation et de construction. Cette garantie devient alors une sécurité pour les donneurs d'ordres qui peuvent évaluer les risques de rénover un bâtiment ou de rechercher une performance énergétique d'excellence. Parce que le travail de donneur d'ordres est un travail à risque, la contractualisation de garantie énergétique peut permettre de sécuriser les opérations incertaines. Ce travail de Contrat de Performance Énergétique (CPE) est traité sous les angles juridique, financier, technique et méthodologique. 

Ce chapitre est donc l'occasion de présenter ce qu'est le \textit{performance gap}, d'analyser les défaillances possibles en phase de conception, chantier, exploitation et enfin de décrire le processus de contractualisation tel que nous le voyons.

\section{Écart de performance}
\label{performance gap}

Dans cette section nous verrons en détail comment les consommations issues des simulations en phase de conception se retrouve à ne pas être égales aux mesures qui dépendent quant à elles aux phases de construction et de exploitation. Avant cela nous proposons ici une revue historique et des projets relatifs à la compréhension des écarts de performance entre théorie et mesures.

Entre 1974, date de la première réglementation thermique, et les années 1990, les bâtiments étaient construits en suivant des exigences de moyens et non pas de résultats. L'atteinte de performance minimum n'était alors pas la préoccupation première de la maîtrise d'ouvrage. Avec le développement des premiers outils numériques de prédiction des consommations énergétiques dans les années 1990, il est devenu possible de fixer des exigences de résultats plutôt que de moyens. En 1994, Norford et al. \cite{Norford-94} ont mis en évidence les différences de résultats entre les modèles et les mesures sur un bâtiment de bureau modélisé avec DOE-2. A cette époque les efforts étaient menés sur la modélisation des systèmes CVC, alors que la prise en compte du bâti était déjà assez fiable. Plus tard, des chercheurs sont parvenus à prendre en considération les phénomènes dynamiques et notamment les apports solaires et sollicitations extérieures. Néanmoins, à cette époque l'attention était peu portée sur les actions des occupants car celles-ci avaient un impact énergétique relativement faible. En effet, l'impact des occupants est directement corrélé avec le niveau de performance du bâtiment. Une modélisation très simplifiée était alors suffisante.

Une plateforme collaborative, CarbonBuzz \footnote{Site internet \url{http://www.carbonbuzz.org/}}, a d'ailleurs été lancée pour recenser les écarts de performance entre théorie et mesure sur des cas d'études. Les résultats montrent généralement des écarts très significatifs, avec des consommations réelles pouvant excéder de 250\% les consommations estimées et une moyenne de dépassement comprise entre 150\% et 200\%. Une mise à jour des outils de prédictions est alors nécessaire pour mieux simuler les bâtiments actuels. Plusieurs groupes de recherche se chargent de réviser les méthodes et les hypothèses actuelles pour réduire ces écarts. 

En 2011, l'Agence Nationale de la Recherche (ANR) a par exemple financé le projet FIABILITE: "Fiabilité des prévisions des performances énergétiques des bâtiments" \footnote{http://www.agence-nationale-recherche.fr/?Projet=ANR-10-HABI-0004} afin de tester la fiabilité des codes de simulation dynamique thermique et énergétique pour les bâtiments à basse consommation (BBC) et à bilan énergétique positif (BEPOS). Un des objectifs primordial et relativement novateur de l'amélioration était d'obtenir des résultats qui reflètent les effets des incertitudes liées aux paramètres de conception et liées aux usages.

En 2013, le projet européen Tribute\footnote{http://www.tipee-project.com/projets/le-projet-tribute} a débuté également dans le but de réduire cet écart. Pour cela l'ensemble des paramètres influents est révisé, avec une attention particulière portée sur le vieillissement des matériaux et des systèmes, les problèmes de santé et la prise en compte des occupants. 

Dans le cadre du programme national PREBAT \footnote{Plateforme de Recherche et d'Experimentation sur l'Energie dans le Bâtiment, Site internet \url{http://www.prebat.net/}}, le CEREMA, supervisé et financé par l'ADEME, a développé une approche permettant d'expliquer les écarts de consommation globale. La méthode proposée pour réduire l'écart, consiste à calculée la consommation des opérations suivant la méthode Th-CE 2005 de la RT 2005 en adaptant les conditions météorologiques, l'occupation et les performances mesurées du bâti et des installations techniques. Ce travail permet alors d'identifier et quantifier finement l'impact de la variation d'un paramètre sur la consommation.

Ces projets et retours d'expériences montrent que les outils de modélisation et de simulation ne sont pas totalement adaptés à la prédiction des consommations énergétiques et qu'une mise à jour est nécessaire pour garantir la performance des opérations. Il est néanmoins sage de rappeler que cette démarche de qualité de la modélisation doit également être menée dans une logique de recherche d'optimum entre un niveau raisonnable d'incertitude et une non sur-qualité. L'outil idéal pour les bureaux d'études est fiable techniquement mais également simple et rapide à utiliser.

\subsection{Conception}

Nous proposons dans cette section d'expliquer en trois phases, pourquoi les simulations ne mènent pas à une reproduction fidèle de la réalité.

La première cause du \textit{performance gap} concerne les hypothèses d'entrée que fixe l'énergéticien. En effet, afin de réaliser une bonne STD il est indispensable de bien connaître son projet et de minimiser le nombre d'hypothèses et d'incertitudes. Pour cela une bonne communication entre les acteurs du projet est indispensable tout comme une excellente maîtrise de l'outil utilisé par le porteur de la simulation. Sans rentrer dans le détail les données d'entrée à considérer sont le climat, l'environnement proche du projet, la géométrie du bâtiment et la définition des zones thermiques, l'enveloppe, les systèmes, les scénarios d'occupations et les paramètres relatifs à la simulation. Pour plus de détails sur la prise en compte de ces paramètres le lecteur pourra se documenter sur Clark \cite{Clarke-01} et Peuportier \cite{Peuportier-16}.

Par nature les logiciels de modélisation n'ont pas les mêmes propriétés les uns des autres, le choix du logiciel, et donc de ses modèles internes, a donc une influence directe sur les résultats. Plusieurs opérations de comparaison inter-logiciels, aussi appelé BESTest\footnote{BESTest pour Balance Evaluation System Test, et non pas Building Energy Simulation Test}, permettent d'évaluer les performances des outils testés. Al-Koussa \cite{AlKoussa-14} propose une comparaison qualitative des logiciels Dymola, Simulink, ESP-r et TRNSYS. Les conclusions reportent des différences significatives pouvant être expliquées par des différences dans le cœur de calcul. Le choix du logiciel utilisé est alors stratégique, le porteur de la simulation doit connaître le champ d'application de son outil pour être en adéquation avec les objectifs de son étude.

La dernière cause d'erreur lors de l'utilisation d'outil de simulation correspond à l'analyse des résultats. Bien que cela commence à changer, les résultats de simulation thermique dynamique sont considérés comme une référence absolue. Or, une évaluation de la pertinence de la valeur obtenue avec un intervalle de confiance prend davantage de sens pour l'analyse post-simulation. En plus de l'évaluation des incertitudes issues des calculs, il est important de traduire les résultats en solutions techniques fiables sur le terrain.

Il est à noter que le travail du simulateur évolue en fonction de l'avancement du projet. Il doit être en mesure d'affiner ses incertitudes et lever certaines hypothèses dans l'objectif ultime de réduire le tunnel d'erreur de la simulation.

\subsection{Construction}

Nous venons de voir que la qualité des études de conception dépend beaucoup des connaissances de l'énergéticien sur le projet et du logiciel utilisé. Néanmoins, ces études ne prennent généralement pas en compte les défauts constructifs, et cela ne doit pas changer selon nous.  En phase de construction il est nécessaire d'être vigilant sur la mise en place des différents éléments constituant le bâtiment et ses systèmes. 

Les professionnels du secteur ont parfois peu de formation sur la mise en œuvre de techniques performantes d'un point de vue énergétique. Les bâtiments performants font appel à des techniques constructives nouvelles, qui ne sont pas toujours parfaitement maîtrisées par les entreprises du bâtiment. Parmi les défauts de construction constatés, il est fréquent d'observer des discontinuités d'isolant, des infiltrations d'air au niveau du réseau électrique et fluide, des pares-vapeur mal installés ou encore des joints d'étanchéité mal posés sur les menuiseries. Cette liste non-exhaustives de défauts de construction est résultante de professionnels pas toujours qualifiés ou appliqués, de désaccords entre les artisans des différents corps d'état ou encore d'une supervision des travaux insuffisante. Ces raisons qui peuvent expliquer une mauvaise qualité du chantier sont à encadrer pour éviter des dérives de consommations. En effet, le suivi continu des étapes du chantier évite que la non-qualité ne se généralise dans le bâtiment. 

Pour encadrer l'évaluation de la qualité sur les chantiers et donc répondre aux besoins des conducteurs de travaux et aux maîtres d'œuvre, des logiciels rendent possible un contrôle qualité des ouvrages. Ces logiciels disponibles sur tablettes numériques et smart-phones facilitent l'identification des problèmes de qualité tout au long du chantier. Ces applications, telles que FinalCAD \footnote{Site internet: \url{http://www.finalcad.com/fr/}, visité le 23/12/2015} ou Air-Bat\footnote{Site internet: \url{http://www.air-bat.fr//}, visité le 23/12/2015} permettent donc d'effectuer les relevés de réserves sur le chantier, puis dans un deuxième temps de contrôler, corriger et générer les procès verbaux. Ce suivi numérique rend alors possible un contrôle qualité exhaustif tout au long du chantier sur la totalité des points clés de l'ouvrage.

La qualité chantier est donc un gage de limitation du \textit{performance gap}, c'est à dire de maîtrise des consommations énergétiques et de confort des occupants.

\subsection{Exploitation}

Une bonne simulation et une parfaite livraison d'un bâtiment n'est pas la certitude, loin de là, d'un faible écart entre performance théorique et réelle. La manière dont l'exploitation est faite par ses usagers, les phénomènes météorologique et la maintenance influent sur les performances réelles.

\subsubsection{Usages et usagers}
\label{Usages et usagers}

Les scientifiques, tels que Hoes \cite{Hoes-09}, Chen \cite{Chen-12}, Kashif \cite{Kashif-13}, et bien d'autres sont en accord unanime pour dire que le comportement des usagers est un des paramètres d'entrée influençant le plus les simulations énergétiques des bâtiments. DeMeester et al. \cite{deMeester-13} précisent que les activités humaines impactent particulièrement les consommations relatives dans le cadre de bâtiments fortement isolés. Degelman \cite{Degelman-99} confirme lui que les simulations énergétiques sont depuis quelques années proches de la perfection, mais cela uniquement si les bâtiments sont utilisés de manière routinière et prévisible, ce qui n'est jamais le cas. Des études de sensibilité sur l'impact du comportement des occupants, tel que l'état de l'art de Larsen et al. \cite{Larsen-10}, attestent de la nécessité d'être rigoureux dans la modélisation des comportements vis à vis de l'énergie. Dans ce document technique, 1000 logements similaires ont été suivis dans la banlieue de Copenhague et après pondération des résultats, les consommations finales d'énergie montrent d'énormes différences dues aux pratiques des occupants. 

Toutes ces études montrent finalement qu'en fonction des usages et usagers, un bâtiment performant peut consommer davantage qu'un bâtiment théoriquement moins performant. Dans les deux sections suivantes, nous proposons de reprendre la définition du comportement des occupants de Zaraket \cite{Zaraket-14} qui distingue une part rationnelle du comportement d'une autre aléatoire.

\paragraph{Part rationnelle}

Dans le contexte du bâtiment, et principalement en résidentiel, les actions des occupants impactant les consommations énergétiques sont nombreuses, tout comme les paramètres influant le comportement des occupants vis à vis de l'énergie. Cette complexe relation entre les occupants et leur environnement est présenté schématiquement dans la Figure \ref{fig:Comportement_occupant_energie}. Ce schéma traduit en français à partir des travaux de l'Annexe 53 \cite{Annex-53-1} de l'Agence Internationale de l'Énergie (IEA), présente et organise l'ensemble des paramètres modifiant les comportements énergétiques. On note deux grandes familles de comportements, d'un coté les paramètres internes (biologique, psychologique et social) qui concernent directement les occupants et de l'autre les paramètres externes liés aux bâtiments et aux conditions environnementales. Les paragraphes suivants illustrent à titre d'exemple en quoi le genre, les interactions de groupe et la facilité de l'opération modifient les consommations énergétiques.

\begin{figure}[H]
\centering
\includegraphics[scale=0.56]{Images/Comportement_occupant_energie}
\caption{Relation entre le comportement des occupants et les consommations d'énergie}
\label{fig:Comportement_occupant_energie}
\end{figure}

% Ex Genre

La Figure \ref{fig:Comportement_occupant_energie} indique que le sexe est un paramètre biologique qui influe sur les consommations énergétiques. En effet, plusieurs études physiologiques ont montré que les femmes et les hommes n'ont pas les mêmes attentes en terme de confort. En effet, Foda et al. \cite{Foda-11} et Jacquot et al. \cite{Jacquot-14} ont montré que les sentions thermales ne sont pas identiques selon le genre, cela modifie donc les températures de consigne et en conséquence les consommations énergétiques. Il a été prouvé que les femmes ont tendance à moins bien supporter le froid que les hommes, alors que ces derniers sont plus vulnérables face aux températures élevées.

% Ex Interactions de groupe

Les interactions de groupe peuvent se traduire par des conflits au sein des ménages. Les rapports entre un mari et sa femme, entre des parents sur leurs enfants ou un patron sur ses employés peuvent influer sur les pratiques énergétiques. Certaines relations sont plutôt démocratiques, d'autres sont plus autoritaires, ainsi des règles de priorité peuvent être établies au sein du groupe social. La présence d'invités au domicile est un moment de surconsommation avec des rituels d'accueil (mise en fonctionnement d'appareils de cuisine, augmentation de la température, augmentation de l'intensité lumineuse, musique, etc.) calqués aux normes sociales mais qui participent également à une certaine mise en scène de l'identité familiale. Une communauté scientifique de sociologues de l'énergie, aujourd'hui encore organisée autours des Journées Internationales de la Sociologie de l'Énergie \footnote{Site officiel des deuxièmes JISE 2015 à Tours: \url{http://www.socio-energie2015.fr}} (JISE) se développe afin de mieux comprendre ces thématiques comportementales. 

% Ex Accessibilité aux systèmes

Plusieurs études ont montré que, lorsque l'on s'intéresse aux actions des occupants dans le bâtiment, l'accessibilité aux systèmes de contrôle modifiait les comportements. Sutter et al. \cite{Sutter-06} ont étudié l'utilisation des stores dans les bureaux en fonction de leur accessibilité. Ils en ont conclu que les stores électriques (accessibles à distance) des bureaux étaient trois fois plus utilisés que les stores manuels. Cette étude a été réalisée en observant de l'extérieur le nombre de montées-descentes des stores. Pour sa part, Andersen \cite{Andersen-09} confirme que l'accessibilité des systèmes de contrôle joue un rôle majeur dans le comportement des usagers et affirme que le temps avant de s'adapter à une zone d'inconfort diminue lorsque les moyens d'adaptation sont nombreux et disponibles. Cette disponibilité des contrôles sur le confort visuel ou thermique modifie alors les comportements adaptatifs des usagers et a en conséquence un impact direct sur le bilan énergétique.

\paragraph{Part aléatoire}

Nous venons de voir que les comportements humains sont rationnels et qu'ils dépendent de paramètres dits "internes" et "externes". En plus de ces relations plutôt rationnelles, il y a également une part d'insaisissable dans l'humain qui ne réagit pas de manière déterministe tel une machine. Un même individu dans un contexte exactement similaire ne réagira alors pas toujours de la même façon. Cette part aléatoire des comportements humains doit ainsi être mieux appréhendée pour l'intégrer aux calculs de performances énergétiques. D'après les travaux d'O'Brien et Gunay \cite{O'Brien-14} et de Leaman \cite{Leaman-99}, nous pouvons pouvons dire que les occupants attendent un certain temps avant de s'adapter lorsqu'ils sont en zone d'inconfort, qu'ils compensent en excès leurs actions à des inconforts mineurs, qu'ils prennent l'option la plus facile et la plus rapide pour un effet immédiat plutôt que la meilleure et qu'ils laissent consciemment ou non les systèmes dans leur état après un changement plutôt que de les réinitialiser après que l'inconfort soit passé. L'ensemble de ces éléments rendent compte de la difficulté de comprendre le processus comportemental. 

\paragraph{Ensemble stochastique}

Comme nous venons de le voir le comportement des occupants peut se décomposer en une part rationnelle et une part aléatoire. La traduction des études sociologiques en informations exploitables est difficile car les données sont davantage qualitatives que quantitatives. De plus, ces données sont souvent très disparates et privées de droits. On peut tout de même recenser certains organismes, comme le Centre de Recherche pour l'Etude et l'Observation des Conditions de vie (CREDOC), qui recense quelques données quantitatives sur la façon dont vivent les habitants d'un point de vue social mais également énergétique. 

Dans le but de synthétiser et de catégoriser les paramètres d'influences sur le chauffage, le refroidissement, la ventilation, les opérations sur les fenêtres, l'utilisation de l'eau chaude sanitaire, de l'usage d'électricité et de l'éclairage, l'Annexe 53 \cite{Annex-53-1} de l'Agence Internationale de l'Energie a créée des tableaux de synthèse. Le Tableau \ref{tab:drivingforce} présente les niveaux d'importance des paramètres influençant les comportements et le Tableau \ref{tab:drivingforceheat} est une application pour les comportements vis à vis du chauffage. Le Tableau se lit comme: la température de consigne dépend fortement du niveau d'isolation du bâtiment et de la température extérieure, alors que la durée du chauffage dépend très peu des intentions gouvernementales. L'ensemble des tableaux se trouve dans le rapport final de l'Annexe 53.

La décomposition du processus comportemental des occupants, avec un composant incertain, démontre alors qu'il n'est pas possible de prédire une performance unique mais plutôt une plage de performance. Cela montre alors que le \textit{performance gap} est largement expliqué par les usagers.

\begin{table}
\begin{center}
\begin{tabular}{|c|c|}
\hline
\multicolumn{2}{|c|}{\textbf{Importance}} \\
\hline
\hline Description & Symboles \\
\hline Très hautement significatif ($p\leq 0.001$) & \cellcolor{OliveGreen} *** \\
\hline Hautement significatif ($p\leq 0.01$) & \cellcolor{LimeGreen} ** \\
\hline Modérément significatif ($p\leq 0.05$) & \cellcolor{yellow} * \\
\hline Peu significatif ($p\leq 0.1$) & \cellcolor{orange} ' \\
\hline Pas significatif & \cellcolor{red} p.s. \\
\hline Pas déclaré & \cellcolor{gray} x \\
\hline
\end{tabular}
\caption{Notation utilisée pour l'importance des paramètres influents; la valeur p correspond au niveau statistique}
\label{tab:drivingforce}
\end{center}
\end{table}

\begin{table}
\begin{center}
\begin{tabular}{|p{2cm}||p{2cm}|p{2cm}|p{2cm}|p{1.5cm}|p{2cm}|p{2cm}|}
\hline & Biologique & Psychologique & Sociologique & Temporel & Environnement physique & Bâtiment et équipements \\
\hline
\hline \multirow {3}{1pt}{Consigne de température} & \cellcolor{gray} Genre \cite{Andersen-09} & \cellcolor{yellow} Attentes \cite{Keul-11} & \cellcolor{LimeGreen} Possession (propriétaire, locataire) \cite{Keul-11} & \cellcolor{orange} Heure de la journée \cite{Andersen-09} & \cellcolor{OliveGreen} Température extérieure de l'air \cite{Larsen-10} & \cellcolor{OliveGreen} Niveau d'isolation du bâtiment \cite{Muller-10} \\
\cline{2-7} & \cellcolor{LimeGreen} Habits \cite{Andersen-09}\cite{Keul-11} & \cellcolor{OliveGreen} Fréquence d'interaction avec le système de contrôle \cite{Andersen-09} &  &  & \cellcolor{LimeGreen} Humidité de l'air extérieure \cite{Andersen-09} & \cellcolor{yellow} Type de ventilation \cite{Keul-11} \\
\cline{2-7} &  & \cellcolor{yellow} Ouverture de fenêtre \cite{Andersen-09} &  &  &  &  \\
\hline \multirow {3}{50pt}{Durée de l'activation du chauffage} & \cellcolor{LimeGreen} Habits \cite{Andersen-09}\cite{Keul-11} & \cellcolor{LimeGreen} Compréhension des fonctions de contrôle \cite{Andersen-09}\cite{Keul-11}\cite{Peeters-08} & \cellcolor{Orange} Possession (propriétaire, locataire) \cite{Andersen-09} &  & \cellcolor{OliveGreen} Température extérieure de l'air \cite{Andersen-09} & \cellcolor{OliveGreen} Niveau d'isolation du bâtiment \cite{Muller-10} \\
\cline{2-7} &  &  &  &  & \cellcolor{OliveGreen} Humidité de l'air extérieure \cite{Andersen-09} & \cellcolor{OliveGreen} Type du système de chauffage \cite{Andersen-09} \\
\cline{2-7} &  & \cellcolor{yellow} Ouverture des fenêtres \cite{Andersen-09} & \cellcolor{red} Intentions gouvernementales \cite{Muller-10} &  & \cellcolor{LimeGreen} Vitesse du vent \cite{Andersen-09} & \cellcolor{OliveGreen} Niveau de contrôle \cite{Andersen-09} \\
\hline Nombre de pièces chauffés &  & \cellcolor{OliveGreen} Fréquence des interactions avec les systèmes de chauffage \cite{Andersen-09} &  &  &  & \cellcolor{OliveGreen} Niveau de contrôle \cite{Andersen-09} \\
\hline Pièces chauffés & \cellcolor{orange} Genre \cite{Andersen-09} &  &  &  &  & \cellcolor{OliveGreen} Niveau de contrôle \cite{Andersen-09} \\
\hline
\end{tabular}
\caption{Valeur des paramètres d'influence sur le chauffage de l'espace selon la littérature}
\label{tab:drivingforceheat}
\end{center}
\end{table}

\subsubsection{Météo}
\label{Météo}

En phase d'exploitation, la météo, en plus des usages et usagers, modifie les performances des bâtiments.  Les variables météo requises pour réaliser des STD sont généralement renseignées au pas de temps horaire dans les fichiers météo. Les variables principales sont la température extérieure, l'humidité, les vents, les précipitations et le rayonnement solaire et sont intégrées dans les calculs de manière dynamique.

La création de ces fichiers météo est standardisée. Le développement se base sur quelques dizaines d'années d'observation météorologiques où les mois les plus représentatifs de ces mesures sont sélectionnés pour constituer le fichier météo. Ainsi, les 12 mois les plus typiques sont choisis individuellement puis les transitions sont lissées afin d'avoir de la continuité dans le fichier. Bien que censé recréer au plus proche la réalité, la prise en compte en compte des incertitudes est un challenge de part la complexité de la météorologie. Les phénomènes de micro-climat, d'îlot de chaleur, de changement climatique et variabilité annuelle témoignent de cette difficulté. Il va donc de sens commun de comprendre que le fichier météo est une source d'incertitude de performance énergétique des bâtiment.

Dans le cadre de la garantie de performance énergétique, il est de rigueur de recaler la simulation sur les mesures sur site afin de ne pas justifier un \textit{performance gap} sur le climat.

\subsubsection{Maintenance}

Cette section s'intéresse à la maintenance des systèmes des bâtiments. En prémisse, on peut rappeler que les consommations énergétiques sont égales aux besoins plus les pertes. Lors d'une approche de modélisation bottom-up, ou dite de synthèse, comme une simulation thermique dynamique, prédire les consommations énergétiques nécessite de connaitre d'une part les besoins mais également les propriétés des systèmes. Une bonne efficacité des systèmes est alors essentielle pour réduire les consommations, tout comme une bonne connaissance des rendements est essentielle pour prédire les consommations réelles. La dégradation des performances des systèmes lors de l'exploitation des bâtiments est alors à considérer pour éviter les dérives de consommations et maîtriser les estimations. Cette dégradation liée à l'usage et à la mise en œuvre des composants est étudiée dans le projet ANR (Agence Nationale de la Recherche) MAEVIA: "Modèles Appliqués à l'Energie et à la Ventilation Interopérables et Adaptables" \footnote{http://www.agence-nationale-recherche.fr/?Projet=ANR-12-VBDU-0005}, du programme VBD (Villes et Bâtiments Durables). Bien que ce projet soit prioritairement appliqué à la qualité de l'air intérieur, la simulation du bâtiment dans ses conditions réelles d'utilisation est tout de même fondamentale. En effet, la dégradation des systèmes peut impacter la Qualité de l'Air Intérieur (QAI) et les performances énergétiques. Cette liaison est l'objet d'étude de MAEVIA, qui développe un outil applicable à la QAI pouvant se coupler à un outil de simulation thermique.

Brisepierre \cite{Brisepierre-11}, sociologue de l'énergie a enquêté sur le rapport entre les équipements et les usagers. Il a alors souligné les difficultés qu'ont les occupants d'une part à piloter leurs équipements et d'autre part à les entretenir. Cela, entraîne des utilisations non optimales qui réduisent les performances et donc augmentent les consommations énergétiques. Brisepierre a noté que la gestion de la ventilation est particulièrement mal gérée par les occupants alors que son impact sur les consommations énergétiques est très fort. Une mauvaise gestion automatique ou manuelle du renouvellement d'air et une bouche d'extraction encombrée de poussière ont un impact significatif sur le confort et les consommations énergétiques. La connaissance des systèmes, de leurs pilotage et de la sensibilisation qu'ont les occupants de ses systèmes sont autant de paramètres à prendre en considération pour l'estimation des consommations énergétiques. 

\section{Engagement performantiel}

Comme les sections précédentes le laisse comprendre, de nombreux paramètres influencent les performances réelles des bâtiments. Estimer des consommations est alors un exercice complexe, mais qui doit être réalisé avec soin pour que les travaux de constructions ou de rénovations atteignent bien les performances visées. En effet, les travaux à hautes performances énergétiques ne peuvent se réaliser à grande échelle que si les commanditaires des travaux de rénovations énergétiques ont la certitude d'obtenir les économies d'énergies vendues et les gains financiers associés.

S'engager dans des travaux de rénovation lourds ou dans des opérations de construction très performantes implique un risque fort pour les donneurs d'ordre. Afin de les rassurer sur l'efficacité réelle de tels travaux il existe des CPE (Contrats de Performance Énergétique), dont la fondation Bâtiment-Énergie \cite{FBE-16} est à l'origine et qui propose un guide d'accompagnement sur l'élaboration d'une méthodologie de GRE (Garantie de Résultats Énergétiques). Ces contrats visent à sécuriser les actions de l'ensemble des acteurs pour garantir des résultats. Ils proposent aux co-contractant de se mettre d'accord sur une GPE (Garantie de Performance Énergétique) couvrant telle ou telle partie de la vie d'un bâtiment. La GRE est la forme la plus complète de GPE de la conception à l'exploitation d'un bâtiment. La GRE garantit au donneur d'ordre que la consommation, après travaux et ajustements éventuels ne dépassera pas une certaine valeur. En cas de non atteinte de la garantie et en fonction des termes de contrat, les responsabilités sont recherchées afin de dédommager le client. Ce travail contractuel concerne les travaux de rénovation mais également les bâtiments neufs, pour lesquels les moindres consommations d'énergie annoncées doivent être sécurisées, en regard des investissements complémentaires consentis.

Les prestataires, par une GRE, mobilisent une partie du gisement d'économies d'énergie qui n'avait pas été exploitée pour des raisons techniques, financières ou organisationnelles par le consommateur final. Engager des travaux de rénovation revalorise le patrimoine et contribue à rembourser l'investissement initial et donc à réduire les coûts d'exploitation.

Cette section présente sous les angles, juridique, financier, technique et méthodologique, la mise en place comme nous l'imaginons d'un engagement contractuel sur la performance énergétique sécurisant et rentable. 

\subsection{Juridique}

L'introduction sur l'engagement contractuel a permis de présenter l'intérêt de développer la contractualisation de performance énergétique car elle permet de mieux concevoir, de mieux réaliser, de mieux exploiter et de mieux communiquer sur les opérations. L'engagement permet d'assurer l'économie du projet pour le maître d'ouvrage et les utilisateurs en répartissant les gains ou en pénalisant les fautifs. 

D'un point de vu juridique, il est essentiel de définir les porteurs de l'engagement. La législation actuelle est souple et laisse la liberté aux co-contractants de négocier leurs limites d'interventions et donc de responsabilités. Quel que soit l'acteur qui porte l'engagement (la maîtrise d'ouvrage, la maîtrise d'œuvre, l'exploitant ou une tierce partie), il possède un droit d'intervention sur l'ensemble des autres acteurs qui ont un lien avec la performance énergétique.

La fondation Bâtiment-Énergie \cite{FBE-16} préconise quatre schémas de contractualisation possible selon le degré d'implication, de capacité et de compétence du donneur d'ordre. Dans le schéma 1, la GRE est portée à 100\% par un prestataire qui intervient dès les premières phases du projet, c'est à dire lors de la phase d'audit pour les travaux de rénovation. Le schéma 2, le plus utilisé, implique davantage le donneur d'ordre en lui confiant l'audit énergétique et la programmation performantiel prévisionnel, un prestataire s'occupe ensuite d'identifier et de tester les gains potentiels puis propose un engagement tenable. Le schéma 3 est semblable au 2 avec cependant un niveau de définition des solutions plus avancé par la maîtrise d'ouvrage. Dans cette configuration, le prestataire se voit réduire sa palette de solutions pour atteindre l'objectif. Le dernier schéma juridique rend la garantie de résultats énergétiques à la maîtrise d'œuvre qui porte l'engagement contrairement aux 3 précédant schémas qui est porté par une entreprise de service énergétique. 

Finalement, le choix du schéma par le donneur d'ordre dépend principalement de ses compétences. Malgré cela, le schéma 1, plus cher car moins impliquant pour le donneur d'ordre, ne l'affranchit pas d'un suivi précis des phases du contrat et d'un accompagnement du prestataire ou du maître d'œuvre.

\subsection{Financier}

L'aspect financier des projets de GPE est fondamental car il dicte les choix des donneurs d'ordres. L'angle financier pris dans cette section ne considère que les sources de financement et deux indicateurs de l'estimation de la rentabilité des projets.

Par essence, un contrat de performance énergétique garantie une réduction des consommations énergétique et donc des dépenses moindre à la suite de projets de rénovation énergétique. Les économies sont évidement effectives qu'après l'achèvement des travaux puis durent pendant toute la phase d'exploitation. Une deuxième source de financement un peu particulière concerne les Certificats d'Economie d'Energie (CEE). A la suite de travaux d'amélioration de performance énergétique, il est possible de demander ces CEE qui sont monnayables auprès d'un fournisseur d'énergie. Après validation, le prix de l'énergie consommé sur le site se retrouve diminué. La troisième source de financement et la plus évidente concerne les fonds propres de l'investisseur. Bien sûr, nous pouvons également considérer la revalorisation patrimoniale et la valeur verte comme valorisant les bâtiments rénovés, cela n'est pas une source de financement à proprement parlé, mais est entièrement considéré par les donneurs d'ordres. Enfin, dans le cas où les fonds propres et autres aides ne sont pas suffisants, les donneurs d'ordres peuvent faire appel à un tiers financement. Le financeur prête alors au donneur d'ordre qui le remboursera en phase d'exploitation en intégrant une partie des économies réalisées grâce aux améliorations énergétiques.

Concernant ce tiers financement des Sociétés Publiques Locales (SPL) comme la SPL d'efficacité énergétique OSER\footnote{Site internet \url{http://spl-oser.fr/}} de la région Rhone-Alpes ou la Société d'Aménagement et d'Equipement de la Région Parisienne (SAERP)\footnote{Site internet \url{http://www.saerp.fr/}} proposent de prendre en charge l'ensemble des étapes des projets avec les collectivité de la région, notamment financier. Ce type de structure en plein essor propose des actions transversales en se faisant déléguer le travail de maîtrise d'ouvrage des collectivités par la signature d'un Bail Emphytéotique Administratif (BEA). Les SPL font appel à des fonds d'investissement et négocient les conventions de prêts, puis négocient les CPE avec les entreprises locales. Cette prestation est dans cet exemple à destination des collectivités locales, mais est également applicables pour assister des maîtres d'ouvrages en copropriétés ou dans le tertiaire.

En plus du financement initial du projet de GRE, le donneur d'ordre doit estimer la rentabilité financière du projet. Pour ce faire, il peut raisonner en rendement brut de l'investissement, c'est à dire en évaluant le rapport économie annuelle suite aux travaux de rénovation sur l'investissement total. Le temps de retour sur investissement est le deuxième indicateur que nous pouvons citer pour évaluer la rentabilité d'un projet, c'est la durée nécessaire pour que l'investissement total soit remboursé grâce aux économies annuelles. Lorsque le donneur d'ordre réalise ce type de calcul il est important qu'il considère les frais financiers (actualisation) et l'évolution prévisionnelle du prix de l'énergie.

\subsection{Technique}
\label{Engagement performantiel - Technique}

Les Contrats de Performance Energétique (CPE) impliquent à certains acteurs de la conception et construction de prévoir les futures consommations. L'Institut Français pour la PErformance des Bâtiments (IFPEB) \cite{IFPEB-2014} propose un guide, dont nous adhérons, construit autours de quatre piliers essentiels pour s'engager sur des futures consommations. Avant de présenter ces piliers, il est important de sensibiliser les donneurs d'ordre sur l'intérêt des démarches intégrées comme les contrats Conception / Réalisation / Exploitation / Maintenance (CREM) qui permettent de réduire l'effet de rupture technique à la livraison, l'exploitant ayant participé activement à la conception puis s'étant engager sur un contrat de performance énergétique.

Le premier pilier et peut être le plus fondamental est de bien définir les usages, l'intensité d'usage et le potentiel d'usage du bâtiment, qu'il soit neuf ou en rénovation. Orienter la conception côté utilisateur est un gage de confort qui a parfois tendance à être délaissé lors de la course à la performance énergétique. Néanmoins, dans le cycle de vie d'un bâtiment plusieurs usages y ont lieu, il faut donc construire des bâtiments robustes qui ne verront pas leurs consommations dériver pour un certain type d'usage.

Le second pilier et celui qui nous intéresse le plus dans ce manuscrit de thèse consiste à prévoir les futures consommations énergétiques tous usages compris. Le premier pilier servant à définir un usage nominal mais également des potentiels d'usages, la Simulation Energétique Dynamique (SED) de l'opération doit également prendre en considération les différents usages du bâtiment. Cette approche est également défendu par Lenormand \cite{Lenormand-15} qui prône une approche des SED par les tangentes plutôt que par les ponts, c'est à dire où l'énergéticien réalise des simulations dans des situations extrêmes qui lui permettent de tester la flexibilité des opérations. Cette approche de simulation permet alors de tester des comportements, scénarios ou situations extrêmes, mais ne permet pas de prédire des performances dans un intervalle de confiance. Pour cela, il est nécessaire de réaliser plusieurs simulations en faisant varier un certain nombre de paramètres afin de réaliser une analyse de l'incertitude. Comme nous l'avons évoqué en introduction, les paramètres les plus incertains lors des simulations des bâtiments sont ceux relatifs aux occupants. Cette incertitude est propre aux comportements des humains, d'où le développement de modèles stochastiques à base d'agents de ce travail de thèse. La simulation multiple permet alors d'obtenir en sortie de SED une densité de probabilités des consommations énergétiques, plutôt qu'une valeur unique qui n'apporte pas d'indication sur la flexibilité intrinsèque du bâtiment.

Le troisième pilier de l'IFPEB, que nous reprenons également ici, concerne la mesure et la vérification de la performance énergétique, communément appelé Plan de Mesure et de Vérification (PMV), ainsi qu'une supervision énergétique. Une fois le bâtiment livré, il faut mesurer les consommations réelles en les ajustant aux conditions météorologiques et d'usages, cela permet de vérifier si les consommations prédites étaient bonnes. Si, c'est le cas alors le contrat est rempli pour le porteur de l'engagement, si ce n'est pas le cas alors une investiture est menée pour comprendre et réparer les défaillances. Ces défaillances peuvent être de diverses natures et peuvent mener soit à de nouveaux travaux, soit à un travail d'Assistance à Maîtrise d'Usage (AMU) auprès des occupants ou soit à des dédommagements si les deux premières solutions sont infructueuses. En cas de non atteinte des performances visées, les co-contractants doivent se référer à leurs Contrat de Performance Energétique pour les dédommagements. Ce pilier est donc nécessaire pour vérifier les clauses du CPE, mais est surtout un atout technique pour améliorer la performance énergétique des bâtiments et atteindre le facteur 4 généralement visé par une réhabilitation énergétique lourde ou une opération neuve exemplaire.

Le dernier pilier consiste à réduire le risque de non atteinte des objectifs sur toute la durée du projet, c'est ce que nous appelons le \textit{commissioning} ou supervision en français. Cette supervision est finalement extérieure au projet et fait la synthèse des trois autres piliers décrits auparavant. Il s'assure donc en début de projet que les exigences du donneur d'ordre en terme énergétique ont été appropriées par la maîtrise d'œuvre. En phase d'études le \textit{commissioning} assiste l'équipe de conception puis assiste les entreprises en phase de travaux. Avant la livraison du bâtiment, il s'assure de la conformité des différents tests, tels les tests d'étanchéité à l'air, puis s'assure de la bonne prise en main du bâtiment par l'exploitant après livraison. Cette prise en main concerne donc d'une part le travail de suivi énergétique dans le cadre du troisième pilier et d'autre part la connaissance des systèmes et des opérations de maintenance à prévoir. 

\subsection{Méthodes}

Le développement de la méthodologie de projets de Garantie de Résultats Energétique (GRE) modifie le schéma de conduite de projet et de contractualisation habituels. Le constat de la méthodologie actuelle montre un écart significatif entre performance théorique et réelle dû à la linéarité des étapes du projet de construction et surtout à la contractualisation habituelle qui fixe des objectifs techniques constructifs et des obligations de moyens. Contrairement à l'approche classique, la méthodologie de la GRE permet une définition plus précise du niveau réel de performance atteignable en considérant les retours du terrain et en adoptant un processus d'affinement du tunnel de risque.

Ce tunnel de risque a vocation à se rétrécir au fur à mesure de l'évolution des phases du projet et de l'identification des améliorations de la performance énergétique en ce qui concerne la rénovation énergétique. Lorsque le choix des travaux ou de bouquet de travaux est définit, l'équipe de conception peut alors prédire une consommation assez finement si l'usage est connu.

Une fois le bâtiment livré avec les améliorations de performance énergétique, la moitié du travail reste à réaliser. La fondation Bâtiment-Énergie \cite{FBE-16} nomme cette seconde étape le processus bouclé en phase d'exploitation. En mesurant la performance ajustée aux conditions réelles d'exploitation du bâtiment et en la comparant à la performance prédite et garantie, et en cas de différence trop importante il doit être possible de corriger l'exploitation ou revoir les travaux réalisés. Cette méthodologie permet ainsi d'atteindre le niveau de performance visé en ajustant les dispositifs ou en accompagnant les occupants sur les bonnes pratiques. Si les ajustements ne sont pas suffisants alors les pénalités juridiques et financières prévues peuvent s'appliquer à l'encontre des fautifs.

La GRE n'est pas le seul type de Contrat de Performance Énergétique (CPE), il existe également la Garantie de Performance Énergétique Intrinsèque (GPEI) qui se limite au stade de la conception et assure la performance intrinsèque du bâtiment et des équipements sans considérer l'usage. Ce type de contrat est moins contraignant pour les donneurs d'ordre et co-contractants car il n'est pas engageant sur la durée d'exploitation. Ce type de contrat bien que de plus en plus utilisé n'est pas présenté dans ce manuscrit, car il ne considère pas les usages: sujet majeur de cette thèse.

Nous pouvons rappeler que l'engagement performantiel est bien une nouvelle méthode de travail pour les concepteurs, mais qui implique des coûts supplémentaires quelque soit le type d'opération. En effet, bien entendu la non-qualité est très chère, mais l'ultra perfection l'est également, un point d'équilibre existe. Le critère de décision principal pour réaliser ou non un projet de GRE repose sur l'équilibre entre ce coût supplémentaire et le gain en termes de réduction du risque lié à l'investissement. L'engagement contractuel sur le résultat est un mécanisme souhaitable en cas de rénovation énergétique lourde ou en cas de construction neuve ambitieuse, pour assurer l'économie du projet pour la maîtrise d'ouvrage et les utilisateurs. 

\section{Synthèse}

En partant du constat que les performances énergétiques mesurées sont souvent éloignés de la théorie, nous avons dans ce chapitre évoqué les facteurs pouvant expliquer ces écarts. Nous avons premièrement soulevé trois points critiques à l'utilisation d'outils de simulation pouvant mener à de mauvaises prédictions énergétiques. Puis nous avons examiné comment la phase de construction pouvait nuire à l'atteinte des objectifs. Enfin, nous avons analysé l'impact de la phase d'exploitation sur les performances réelles. 

Le comportement des occupants a été identifié comme un paramètre essentiel mais aujourd'hui mal pris en compte dans les simulation dynamiques. En effet, les méthodes de simulations actuelles utilisent souvent des scénarios hebdomadaires répétitifs pour simuler l'action des usagers sur le bâti. La prise en compte des apports internes, les positions des protections solaires, la régulation des aérations sont des exemples décorrélés de la notion de confort. Une nouvelle approche doit permettre de développer un modèle, dans lequel les utilisateurs ne sont plus aussi prévisibles et passifs, mais interagissent entre eux et sur leur milieu, modifiant certains paramètres de la simulation en fonction de leurs besoins et activités réelles. Ainsi, les phénomènes d'actions et réactions entre les utilisateurs et leur environnement, pourront mieux être intégrés aux simulations numériques. La prise en compte de ces phénomènes permettra entres autres de ne plus obtenir comme résultat d'une étude une consommation énergétique absolue, mais plutôt une fourchette basse et haute permettant de connaître la sensibilité du bâtiment à une utilisation plus ou moins vertueuse de celui-ci.

Enfin, le CPE est un véritable levier d'amélioration de l'efficacité énergétique. Sans dispositif de garantie de résultat énergétique, les objectifs de réduction des consommations d'énergie et des émissions de gaz à effet de serre des bâtiments risquent de ne pas être atteints. Le CPE doit conduire à ce meilleur respect des objectifs annoncés. Défini par une directive du Parlement européen, il permet de mobiliser des ressources humaines et financières au service d'un projet avec une obligation de résultat. Il répond donc à des attentes fortes des acteurs du marché en matière de Garantie de performance énergétique.