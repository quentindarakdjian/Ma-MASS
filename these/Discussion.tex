\chapter{Discussion}

\section{La problématisation avant la modélisation}

Avant de modéliser un phénomène il faut le problématiser. Cela renvoi à des démarches intellectuelles différentes et d'orientation opposées. Aujourd'hui la modélisation est l'approche communément utilisée par les ingénieurs et la problématisation est plutôt utilisée en sciences humaines et sociales.

2009 - Michelot Christian - Modélisation ou problématisation?

\section{Appropriation de la technologie}

Suite à l'apparition des préoccupations environnementales, les bâtiments sont de plus en plus performants, mais aussi de plus en plus standardisés, car soumis à des réglementations toujours plus exigeantes. Pour répondre à ces exigences, les acteurs du bâtiment n'ont pas eu d'autres choix que d'y intégrer les nouvelles technologies et les habitants n'ont pas eu le choix de les assimiler. Or d'après Beslay \cite{Beslay-08}, s'approprier, c'est transformer. En effet, les bâtiments récents sont exigent en terme d'usage et impliquent de nouvelles habitudes, tout en étant devenus difficiles à régler, à exploiter et à habiter. Cette difficulté d'utilisation des bâtiments performants, résulte alors parfois en des comportements déviant voir absurdes qu'il faut corriger pour atteindre les objectifs environnementaux mais également de confort.

\section{L'accompagnement des usagers}

Modéliser le comportement des usagers des bâtiments, ne va pas résoudre seul le problème de performances énergétiques décevantes. Cela va permettre de mieux comprendre les usages et de mieux les appréhender lors du processus de conception ou rénovation des bâtiments. Cette nouvelle connaissance doit en retour profiter aux occupants, par leur accompagnement. Cet activité, généralement sous le nom d'Assistance à Maitrise d'Usage, est en plein essor. 

En région Rhône-Alpes, Vie to B\footnote{Site officiel de Vie to B: \url{http://vie-to-b.fr/}, visité le 23/02/2016} est spécialisé dans l'accompagnement de l'usage de bâtiments performants. Ils ont pour missions principales d'accompagner les occupants dans l'appropriation de leur logement et de les accompagner dans la conciliation entre confort et performance énergétique.

Brisepierre et al. \cite{Brisepierre-15}, dans l'ouvrage "l'accompagnement des occupants: Une évidence à déconstruire", démontrent tout l'intérêt  ...

Lenormand et al. \cite{Lenormand-15}, lors des Journées Internationales de la Sociologie de l'Energie, ont remis en cause les pratiques des énergéticiens qu'ils jugent décorélées des usages quotidiens.


\section{Conclusion}

Nous avons vu dans les chapitres précédents que la sociologie est une science qui permet d'aider au développement de modèles physiques, d'une part dans l'élaboration de questionnaires, d'interviews ou de suivi de mesures mais également dans le traitement des résultats et de l'interprétation qui peut en être faite.

Ce chapitre démontre également que la seule modélisation ne peut pas résoudre les défaillances d'usages, mais qu'un accompagnement personnalisé des usagers est en revanche une solution fiable et durable à mettre en avant dans les prochaines opérations neuves ou de rénovations.